%Abstract
\thispagestyle{plain}
\newgeometry{left=4.5cm,right=4.5cm}
\pagenumbering{gobble}
\vspace*{3cm}
\begin{center}
    \textsc{\LARGE Abstract}

    \rule{5cm}{0.4pt}

    \vspace{1cm}
\end{center}
% Write your abstract
%Background
The path integral approach is a powerful quantization method in quantum field theory.
Through the BV formalism, we are able to compute path integrals considering the gauge symmetries of the problem.
Its extension to BV-BFV formalism allows also for manifolds with boundary.
Thus, we could build a complex and complete spacetime manifold by cutting it into smaller and easier parts, quantizing the space of fields on them and then gluing them back together.
%In particular
This goal is addressed by Cattaneo \etal \cite{Gluing_BV-BFV}.
In the paper, they discuss the details of this procedure.
In particular, they present a theorem and prove it for some special cases.
%What we will do in the thesis
In this thesis, we are going to briefly introduce the necessary background of differential geometry and supergeometry.
We will then study the paper, discuss its main results and prove the theorem for the first two cases, i.e. abelian Chern-Simons theory and abelian BF theory.

\restoregeometry
\newpage