\section{Weak Equivalences}
\label{sec:weak_equiv}

\subsection{Differential Graded Manifold Case}
\label{subsec:dg_mnfs_case}

Following the first section of \cite{Gluing_BV-BFV}, we now introduce the concept of weak equivalence between symplectic manifolds.
Then we expand this notion to the relative case, where the manifolds are allowed to have boundaries.

\begin{definition}
\label{def:weak_1}
    A \emph{weak equivalence} of two $k$-symplectic manifolds $(\mathcal{F}, Q, \omega)$ and $(\widetilde{\mathcal{F}}, \widetilde{Q}, \widetilde{\omega})$ is a pair of maps $f:\mathcal{F} \rightarrow \widetilde{\mathcal{F}}$ and $f:\widetilde{\mathcal{F}} \rightarrow \mathcal{F}$ satisfying the following:
    \begin{itemize}
        \item the pullbacks $f^*:C^\infty(\widetilde{\mathcal{F}}) \rightarrow C^\infty(\mathcal{F})$ and $g^*:C^\infty(\mathcal{F}) \rightarrow C^\infty(\widetilde{\mathcal{F}})$ are quasi-isomorphisms of chain complexes;
        \item it holds that $f^* \widetilde{\omega} = \omega + L_Q \beta$ and $g^* \omega = \widetilde{\omega} + L_{\widetilde{Q}} \widetilde{\beta}$ for some closed $2$-forms $\beta \in \Omega^2 (\mathcal{F})$ and $\widetilde{\beta} \in \Omega^2 (\widetilde{\mathcal{F}})$.
    \end{itemize}
\end{definition}


We can generalize \Cref{def:weak_1} to the case where two symplectic manifolds are both relative to the same manifold, which has the role of common boundary between them.

\begin{definition}
\label{def:weak_relative}
    We define a \emph{weak equivalence} between two $k$-symplectic manifolds $(\mathcal{F}, Q, \omega, \pi)$ and $(\widetilde{\mathcal{F}}, \widetilde{Q}, \widetilde{\omega}, \widetilde{\pi})$, both relative to the same $(k+1)$-symplectic manifold $(\mathcal{F}_\Sigma, Q_\Sigma, \omega_\Sigma)$, as in \Cref{def:weak_1}, requiring additionally that $\widetilde{\pi} \circ f = \pi$ and $\pi \circ g = \widetilde{\pi}$.
\end{definition}

\begin{figure}
    \centering
    \begin{tikzcd}[sep = 8 ex,  every label/.append style={font=\normalsize}]
      \mathcal{F} \arrow[rr, "f", bend left = 15] \ar[dr, "\pi", bend left = 0]
    & 
    & \widetilde{\mathcal{F}} \ar[ll, "g", bend left = 15] \ar[dl, "\widetilde{\pi}", bend left = 0] \\
    & \mathcal{F}_\Sigma
    &
\end{tikzcd}
    \caption{Commutative diagram displaying the additional requirement of \Cref{def:weak_relative}}
    \label{fig:weak_equiv}
\end{figure}

In the above definition, the added requirement makes sure that the projections of fields on the two manifolds $\mathcal{F}$ and $\widetilde{\mathcal{F}}$ onto the manifold $\mathcal{F}_\Sigma$ is the same, in order to consider the manifold $\mathcal{F}_\Sigma$ as a common boundary.

\subsection{Poincaré Complexes Case}
\label{subsec:Poincaré_complex_case}

Finally, we adapt the definitions of the previous subsection to the linear algebra case, by introducing \emph{Poincaré cochain complexes} and the meaning of weak equivalence between them.

\begin{definition}
    A \emph{degree $k$ Poincaré complex} is a cochain complex $\mathcal{F}^\bullet$ over $\mathbb{R}$ with a differential $\dd_Q$ and a graded skew-symmetric non-degenerate pairing $\omega : \bigoplus_i \mathcal{F}^i \otimes \mathcal{F}^{-i-k} \rightarrow \mathbb{R}$, satisfying
    \begin{equation*}
        \omega(\dd_Q x, y) = (-1)^{|x|} \omega (x, \dd_Q y) \qquad \forall x, y \in \mathcal{F} .
    \end{equation*}
\end{definition}

It is possible to generalize \Cref{def:weak_1} to the case of Poincaré complexes, as done below.

\begin{definition}
\label{def:weak_eq_Poincaré}
    A \emph{weak equivalence} between two degree $k$ Poincaré complexes $(\mathcal{F}, \dd_Q, \omega)$ and $(\widetilde{\mathcal{F}}, \widetilde{\dd}_Q, \widetilde{\omega})$ is the following set of data:
    \begin{enumerate}[label=\roman*)]
        \item A pair of chain maps $f:\mathcal{F} \rightarrow \widetilde{\F}$, $g:\widetilde{\mathcal{F}} \rightarrow \F$;
        \item A pair of maps $H: \mathcal{F}^\bullet \rightarrow \mathcal{F}^{\bullet -1}$, $\widetilde{H}:\widetilde{\mathcal{F}}^\bullet \rightarrow \widetilde{\mathcal{F}}^{\bullet -1}$ satisfying the \emph{chain homotopy property}
        \begin{equation}
            \dd_Q H + H \dd_Q = \id -gf \qquad \wt{\dd}_Q \wt{H} + \wt{H} \wt{\dd}_Q = \id -fg ;
        \end{equation}
        \item Two degree $k-1$ skew-symmetric bilinear forms $\beta$ and $\wt{\beta}$ on \F and $\wt{\F}$ respectively, such that
        \begin{align*}
            \wt{\omega}(f(x), f(y)) =&
            \; \omega (x, y) +
            (L_Q \beta) (x, y),
            \quad x, y \in \F \\[6pt]
            \omega(g(\wt{x}), g(\wt{y})) =&
            \; \wt{\omega}(\wt{x}, \wt{y}) +
            (L_{\wt{Q}} \wt{\beta})(\wt{x}, \wt{y}),
            \quad \wt{x}, \wt{y} \in \wt{\F}.
        \end{align*}
    \end{enumerate}
\end{definition}

