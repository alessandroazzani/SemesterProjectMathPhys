\section{Weak Equivalences of dg Manifolds}
\label{sec:weak_equiv}

Following the first section of \cite{Gluing_BV-BFV}, we now introduce the concept of weak equivalence between symplectic manifolds.
Then we expand this notion to the relative case, where the manifolds are allowed to have boundaries.
Finally we adapt the definitions to the linear algebra case, by introducing \emph{Poincaré cochain complexes} and the meaning of weak equivalence between them.

\begin{definition}
\label{def:weak_1}
    A \emph{weak equivalence} of two $k$-symplectic manifolds $(\mathcal{F}, Q, \omega)$ and $(\widetilde{\mathcal{F}}, \widetilde{Q}, \widetilde{\omega})$ is a pair of maps $f:\mathcal{F} \rightarrow \widetilde{\mathcal{F}}$ and $f:\widetilde{\mathcal{F}} \rightarrow \mathcal{F}$ satisfying the following:
    \begin{itemize}
        \item the pullbacks $f^*:C^\infty(\widetilde{\mathcal{F}}) \rightarrow C^\infty(\mathcal{F})$ and $g^*:C^\infty(\mathcal{F}) \rightarrow C^\infty(\widetilde{\mathcal{F}})$ are quasi-isomorphisms of chain complexes;
        \item it holds that $f^* \widetilde{\omega} = \omega + L_Q \beta$ and $g^* \omega = \widetilde{\omega} + L_{\widetilde{Q}} \widetilde{\beta}$ for some closed $2$-forms $\beta \in \Omega^2 (\mathcal{F})$ and $\widetilde{\beta} \in \Omega^2 (\widetilde{\mathcal{F}})$.
    \end{itemize}
\end{definition}


We can generalize \Cref{def:weak_1} to the case where two symplectic manifolds are both relative to the same manifold, which has the role of common boundary between them.

\begin{definition}
\label{def:weak_relative}
    We define a \emph{weak equivalence} between two $k$-symplectic manifolds $(\mathcal{F}, Q, \omega, \pi)$ and $(\widetilde{\mathcal{F}}, \widetilde{Q}, \widetilde{\omega}, \widetilde{\pi})$, both relative to the same $(k+1)$-symplectic manifold $(\mathcal{F}_\Sigma, Q_\Sigma, \omega_\Sigma)$, as in \Cref{def:weak_1}, requiring additionally that $\widetilde{\pi} \circ f = \pi$ and $\pi \circ g = \widetilde{\pi}$.
\end{definition}

\begin{figure}
    \centering
        \begin{tikzcd}[sep = 8 ex, font=\normalsize]
              \mathcal{F} \arrow[rr, "f", bend left = 15] \ar[dr, "\pi", bend left = 0]
            & 
            & \widetilde{\mathcal{F}} \ar[ll, "g", bend left = 15] \ar[dl, "\widetilde{\pi}", bend left = 0] \\
            & \mathcal{F}_\Sigma
            &
        \end{tikzcd}
    \caption{Commutative diagram displaying the additional requirement of \Cref{def:weak_relative}}
    \label{fig:enter-label}
\end{figure}