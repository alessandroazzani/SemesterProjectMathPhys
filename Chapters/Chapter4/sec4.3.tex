\section{Abelian Chern-Simons and BF Theories}
\label{sec:abelain_chern_simons_BF_theories}

In this section we are finally going to prove \Cref{theor:gluing}, first for an abelian Chern-Simons theory and then for an abelian BF theory.

In the proofs, we will need a lemma presented in the paper \cite{Gluing_BV-BFV}, stating as follows.

\begin{lemma}
\label{lemma:tubular_neighbourhood}
    Let $\N$ be an oriented $n$-manifold and $V \subset \N \times \N$ a tubular neighbourhood of the diagonal $\diag \subset \N \times \N$ with projection $r: V \rightarrow \diag$.
    Then:
    \begin{enumerate}[label={\normalfont \alph*)}]
        \item \label{enum:tubular_a} There exists a smooth closed $n$-form $\rho$ on $\N \times \N$ satisfying
        \begin{itemize}[label= $\diamond$]
            \item $\text{\normalfont supp}(\rho) \subset V$,
            \item $r_* \rho = 1$.
        \end{itemize}
        \item \label{enum:tubular_b} Denote $R: \Omega^\bullet_{\distr} (\N) \rightarrow \Omega^\bullet (\N)$ the integral operator from distributional to smooth forms determined by the integral kernel $\rho$.
        Then:
        \begin{itemize}[label= $\diamond$]
            \item $R$ is a chain map w.r.t. the de Rham differential,
            \item Let $\iota: \Omega^\bullet (N) \hookrightarrow \Omega^\bullet_{\distr} (N)$ be the canonical inclusion of smooth forms into distributional forms.
            Then $\iota R$ and $R \iota$ are chain-homotopic to identity.
            More precisely, there exists a distributional form $\chi \in \Omega_{\distr}^{n-1} (\N \times \N)$ supported in $V$ and smooth away from $\diag$ such that $\dd_\chi = \delta_{\diag} - \rho$.
            As a consequence, one has
            \begin{equation}
            \label{eq:chain_homotopy_lemma}
                \begin{aligned}
                    {\normalfont \id}_{\Omega^\bullet_{\distr}} - \iota R =& \; \dd k + k \dd , \\
                    {\normalfont \id}_{\Omega^\bullet} - R \iota =& \; \dd \kappa|_{\Omega^\bullet} + \kappa|_{\Omega^\bullet} \dd .
                \end{aligned}
            \end{equation}
            Here $\kappa: \Omega_{\distr}^{\bullet} (\N) \rightarrow \Omega_{\distr}^{\bullet - 1} (\N)$ is the operator defined by the integral kernel $\chi$.
        \end{itemize}
    \end{enumerate}
\end{lemma}

For the proof of the lemma we refer back to the paper by Cattaneo \etal and to \cite{Proof_second_lemma}.

\subsubsection{Abelian Chern-Simons Theory}
\label{subsubsec:abelian_chern_simons_theory}

The abelian Chern-Simons theory in the BV-BFV formalism (see \cite{Mnfd_boundaries} for details) has $\R[1]$ as target space.
The \emph{bulk BV theory} is composed by a 3-dimensional spacetime manifold $\M$, on which it is defined the space of fields
\begin{equation*}
    \F_\M = \Omega^\bullet (\M)[1],
\end{equation*}
that is the degree shifted de Rham complex.
We denote by $\mc{A} \in \F_\M$ the BV superfield.
It then assigns to $\M$:
\begin{itemize}
    \item A quadratic BV action $\mc{S}_\M = \frac{1}{2}\int_\M \A \wedge \dd \A$;
    \item A $(-1)$-symplectic form $\omega_\M = - \frac{1}{2} \int_\M \delta \A \wedge \delta \A$;
    \item A linear cohomological vector field $Q_\M = \int_\M \dd \A \: \frac{\delta}{\delta \A}$.
\end{itemize}

The bulk manifold $\M$ is cut into $\M_1$ and $\M_2$ by a BFV manifold $\Sigma$.
The boundary fields
\begin{equation*}
    \F_\Sigma = \Omega^\bullet (\Sigma)[1]
\end{equation*}
are pullbacks of the bulk fields to the boundary. The BFV superfield will be denoted with $\A_\Sigma \in \F_\Sigma$.
To the surface $\Sigma$, the theory assigns:
\begin{itemize} % [label= $\circ$]
    \item An Hamiltonian $\mc{S}_\Sigma = \frac{1}{2}\int_\M \A_\Sigma \wedge \dd \A_\Sigma$;
    \item A symplectic form $\omega_\Sigma = \delta \left( \frac{1}{2} \int_\M \A_\Sigma \wedge \delta \A_\Sigma \right)$;
    \item A cohomological vector field $Q_\Sigma = \int_\M \dd \A_\Sigma \: \frac{\delta}{\delta \A_\Sigma}$.
\end{itemize}

The two projections $\pi_{1,2} : \F_{\M_{1,2}} \rightarrow \F_\Sigma$ are the pullbacks from the bulks to the boundary.

The canonical inclusion~(\ref{eq:inclusion}) becomes
\begin{equation*}
    i : \underbrace{\Omega^\bullet (\M)[1]}_{\F_\M} \hookrightarrow 
    \underbrace{\Omega^\bullet (\M_1)[1] \times_{\Omega^\bullet (\Sigma)[1]} \Omega^\bullet (\M_2)[1]}_{\FMS = \F_{\M_1} \times_{\F_\Sigma} \F_{\M_2}} .
\end{equation*}

We now construct the quasi-inverse map $p: \FMS \rightarrow \F_\M$ of the canonical inclusion $i$.
We define an embedding $\psi: \nor \Sigma \rightarrow \M$ of the normal bundle of $\Sigma$\footnote{The \emph{normal bundle} of $\Sigma$ is complementary to the tangent bundle $T\Sigma$ and given by 
$N\Sigma \coloneqq \{ (q, p) | q \in \Sigma, p \in T_q \M / T_q \Sigma \}$.}.
The image $U = \psi(\nor \Sigma)$ is an open neighbourhood of $\Sigma$ which can be made as thin as needed.
If we now apply \Cref{lemma:tubular_neighbourhood} to $\nor \Sigma$, we get from Items~\ref{enum:tubular_a} and \ref{enum:tubular_b} the smooth $n$-form $\rho$ and the distributional $(n-1)$-form $\chi$, respectively.

From these forms, we define a distributional $n$-form $\hat{\rho}$ on $\M \times \M$ by
\begin{equation*}
    \hat{\rho} = 
        \begin{cases}
            (\psi^{-1} \times \psi^{-1})^* \rho & \text{ inside } \,  U \times U \\
             \hfill \delta_{\diag_\M}           &\text{outside } U \times U
        \end{cases}
\end{equation*}
and a distributional $(n-1)$-form $\hat{\chi}$ by
\begin{equation*}
    \hat{\chi} = 
        \begin{cases}
            (\psi \times \psi)_* \chi & \text{ inside } \,  U \times U \\
             \hfill 0                 &\text{outside } U \times U .
        \end{cases}
\end{equation*}

At this point, we can define the quasi-inverse map $p: \FMS \rightarrow \F$ as the integral operator with kernel $\hat{\rho}$.
Notice that $p$ is indeed a chain map, since the kernel $\hat{\rho}$ is a closed form on $\M \times \M$.

Similarly, we define the chain homotopy $\wt{H} : \FMS \rightarrow \FMS[1]$ as the integral operator with kernel $\hat{\chi}$.
If we restrict $\wt{H}$ to $\F_\M$, we get the other chain homotopy $H$.
Notice that Equations~(\ref{eq:chain_homotopy_lemma}) hold the chain homotopy identities in \Cref{enum:theor_b} of \Cref{theor:gluing}.

Thus, we have constructed all of the four maps $(i, p, H, \wt{H})$ satisfying all points of \Cref{theor:gluing}.

\subsubsection{Abelian BF theory}
\label{subsubsec:abelain_bf_theory}

The proof for the abelian BF theory is a direct modification of the construction made for the abelian Chern-Simons theory.
This time, the spacetime manifold $\M$ can be of generic dimension $n$ (not just 3).
In this generalized theory, the space of fields is defined as
\begin{equation*}
    \F_\M = 
    \Omega^\bullet (\M)[\p] \oplus
    \Omega^\bullet (\M)[n-\p-1],
\end{equation*}
where $\p \in \mathbb{Z}$ is some fixed shift and we have two BF superfields $(\mc{A}, \mc{B}) \in \F_\M$.
Similarly, the theory assigns to the spacetime manifold $\M$:
\begin{itemize}
    \item The BV action $\mc{S}_\M = \int_\M \mc{B} \wedge \dd \A$;
    \item The symplectic form $\omega_\M = (-1)^n  \int_\M \delta \mc{B} \wedge \delta \A$;
    \item The differential $\dd_Q = \dd \oplus \dd$.
\end{itemize}

The space of fields on the boundary $\Sigma$ is
\begin{equation*}
    \F_\Sigma = 
    \Omega^\bullet (\Sigma)[\p] \oplus
    \Omega^\bullet (\Sigma)[n-\p-1]
\end{equation*}
and the projections $\pi_{1,2}$ are the pullbacks of $\mc{A}$- and $\mc{B}$-forms from the cut manifolds $\M_{1, 2}$ to $\Sigma$:
\begin{equation*}
    \pi_{1, 2} = \F_{\M_{1, 2}} \rightarrow \F_\Sigma .
\end{equation*}

The canonical inclusion $i$ becomes
\begin{equation*}
    \begin{tikzcd}
    & \overbrace{\Omega^\bullet (\M)[\p] \oplus
    \Omega^\bullet (\M)[n-\p-1]}^{\F_\M}
    \arrow[hookrightarrow]{d}{i} \\
    & \underbrace{(\Omega^\bullet (\M_1) \times_{\Omega\bullet (\Sigma)} \Omega^\bullet (\M_2))[\p] \oplus
    (\Omega^\bullet (\M_1) \times_{\Omega\bullet (\Sigma)} \Omega^\bullet (\M_2))[n-\p-1]}_{\FMS}
\end{tikzcd}
\end{equation*}

Similarly to Chern-Simons theory, we construct the map $p: \FMS \rightarrow \F_\M$ as the integral operator with kernel $\hat{\rho}$ acting on the $\mc{A}$- and $\mc{B}$-forms in $\FMS$.
The homotopies $\wt{H}$ and $H$ are the integral operators wth kernel $\hat{\chi}$ and its restriction to $\F_\M$, respectively.