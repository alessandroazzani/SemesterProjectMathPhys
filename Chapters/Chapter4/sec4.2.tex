\section{Gluing via Fiber Products}
\label{sec:gluing_fiber_prod}

In this section we finally see how to glue together two BV-BFV manifolds.
This is done through \emph{fiber products}, which ensure that a field on the new product space is made of two field, one on each one of the original spaces, such that their projection on the common boundary is the same.

To simplify the introduction of the notion of fiber product, we consider the two manifolds as the result of cutting a bigger manifold into two manifolds with boundaries.
In particular, assume that the original manifold is a closed $n$-dimensional BV-BFV manifold $(\F_\M, \omega_\M, \mathcal{S}_\M)$, denoted by \M.
Here the assumption that $\M$ is closed is not necessary, it is made mostly for simplicity.

Let $\M$ be cut by a closed $(n-1)$-submanifold $\Sigma$ into two $n$-dimensional manifolds with boundaries (not closed anymore) $\M_1$ and $\M_2$.
The original space of fields $\F_\M$ on \M will be cut into two spaces of fields $\F_{\M_1}$ and $\F_{\M_2}$ on the manifolds $\M_1$ and $\M_2$, respectively.

We can now define the fiber product between $\F_{\M_1}$ and $\F_{\M_2}$.

\begin{definition}
    The \emph{fiber product} between two spaces of fields $\F_{\M_1}$ and $\F_{\M_2}$ over the phase space for the interface $\Sigma$ as described above is defined as
    \begin{equation}
    \label{eq:fiber_prod}
        \wt{\F}_{\M, \Sigma} = \F_{\M_1} \fp{\Phi_\Sigma} \F_{\M_2} 
        \coloneqq \{(\phi_1, \phi_2) \in \F_1 \times \F_2 \text{  s.t.}
        \pi_1(\phi_1) = \pi_2(\phi_2) \} .
    \end{equation}
\end{definition}

In the definition above, $\phi_1$ and $\phi_2$ represents two fields on $\M_1$ and $\M_2$, while the maps $\pi_1$ and $\pi_2$ are short terms for what we used to denote as $\pi_{\M_1, \Sigma}: \F_{\M_1} \rightarrow \F_\Sigma$ and $\pi_{\M_2, \Sigma}: \F_{\M_2} \rightarrow \F_\Sigma$ in \Cref{def:BV-BFV}, respectively.
They are the restriction of fields to the boundary.

If we denote the projections of $\M$ into the two manifolds $\M_1$ and $\M_2$ by
\begin{equation*}
    p_1: \wt{\F}_{\M, \Sigma} \rightarrow \F_{\M_1} 
    \qquad
    p_2: \wt{\F}_{\M, \Sigma} \rightarrow \F_{\M_2},
\end{equation*}
then we can also say that the fiber product has to satisfy the following commutative diagram:
\begin{equation*}
    \begin{tikzcd}
    & \wt{\F}_{\M, \Sigma} \arrow[r, "p_2"] \arrow[d, "p_1"]
    & \F_{\M_2} \arrow[d, "\pi_2"] \\
    & \F_{\M_2} \arrow[r, "\pi_1"]
    & \F_\Sigma
\end{tikzcd}
\end{equation*}
Thus, it should be possible to reach the same space of fields on the boundary by considering a field $\phi_1 \in \F_{\M_1}$ and taking its projection on $\Sigma$, or by using the corresponding field $\phi_2 \in \F_{\M_2}$.

Let $\omega_\M$ be the $(-1)$-symplectic structure on $\F_\M$, the $(-1)$-symplectic structure $\wt{\omega}_{\M,\Sigma}$ on $\F_{\M,\Sigma}$ can be written as
\begin{equation}
\label{eq:wt_omega}
    \wt{\omega}_{\M,\Sigma} = 
    p_1^* \omega_{\M_1} + p_2^* \omega_{\M_2} ,
\end{equation}
where $p_{1,2}^*$ denote the pullback by $p_{1,2}$.

Note also that a natural inclusion follows from the definition of fiber product:
\begin{equation}
\label{eq:inclusion}
    i : \F_\M \hookrightarrow \wt{\F}_{\M,\Sigma} .
\end{equation}

\begin{conj}
\label{conj:first}
    The inclusion~\ref{eq:inclusion} can be extended to a weak equivalence of $(-1)$-symplectic manifolds, in the sense of \Cref{def:weak_1}.
\end{conj}

In case of a free BV-BFV theory, Conjecture~\ref{conj:first} can be adapted to the following.

\begin{conj}
\label{conj:second}
    If $\mathcal{T}$ is a free BV-BFV theory, the inclusion~\ref{eq:inclusion} can be extended to a weak equivalence between degree $-1$ Poincaré complexes, in the sense of \Cref{def:weak_eq_Poincaré}.
\end{conj}

In the paper \cite{Gluing_BV-BFV}, they prove the conjectures above for three different BV-BFV theories:
\begin{enumerate}[label={\arabic*)}]
    \item abelian Chern-Simons theory;
    \item abelian BF theory;
    \item $\mathbf{p}$-form electrodynamics in the first- or second-order formalism (including massless scalar field and usual electrodynamics as $\mathbf{P} = 0$ and $\mathbf{p} = 1$ cases).
\end{enumerate}

In this report, we are going to prove it just for the first case, that is for an abelian Chern-Simons theory.
In particular, the proof of the following theorem will be carried out later on, in \textcolor{red}{cite}.

\begin{theorem}
\label{theor:gluing}
    Let $\mathcal{T}$ be an abelian Chern-Simons theory.
    Then, given any open neighborhood $U \subset \M$ of $\Sigma$, one can extend the inclusion~\ref{eq:inclusion} to a package of maps $(i, p, H, \wt{H})$ defined as
    \begin{equation}
    \label{eq:theorem_maps}
        \begin{tikzcd}  
    \F_\M \arrow[r,yshift=1mm, "i"]
    \arrow[loop left, distance=20, out=200,in=160, "H"] 
    & \wt{\F}_{\M, \Sigma} \arrow[l, yshift=-1mm, "p"]
    \arrow[loop right, distance=20,out=18,in=-18, "\wt{H}"]
\end{tikzcd},
    \end{equation}
    such that:
    \begin{enumerate}[label={\textup{\alph*)}}]
        \item \label{enum:theor_a} $i$ and $p$ are chain maps w.r.t. differentials $\dd_Q$ on $\FM$ and $\wt{\dd}_Q$ on $\FMS$ (as illustrated in \Cref{fig:chain_map_theorem}):
        \begin{equation}
        \label{eq:theor_a}
            \begin{aligned}
                \wt{\dd}_Q \; i =& \,\, i \; \dd_Q , \\
                \dd_Q \; p =& \,\, p \; \wt{\dd}_Q;
            \end{aligned}
        \end{equation}
        \item \label{enum:theor_b} H is the chain homotopy between identity and $p \, i$ on $\FM$ and $\wt{H}$ is the chain homotopy between identity and $i \, p$ on $\FMS$:
        \begin{equation}
        \label{eq:theo_b}
            \begin{aligned}
                \dd_Q H + H \dd_Q =&
                \; \id - p \, i , \\
                \wt{\dd}_Q \wt{H} + \wt{H} \wt{\dd}_Q =&
                \; \id - p \, i ;
            \end{aligned}
        \end{equation}
        \item \label{enum:theor_c} the package of maps $(i, p, H, \wt{H})$ is local near $\Sigma$:
        $p$ is smoothing in the neighborhood $U$ of $\Sigma$ and is identity outside $U$. The operators $H$, $\wt{H}$ vanish on fields supported outside $U$;
        \item \label{enum:theor_d} the $(-1)$-symplectic structures on $\FM$ and $\FMS$ are related by
        \begin{align}
        \label{eq:requ_lemma}
            i^* \wt{\omega} =& \; \omega , \\
            p^* \omega =& \; \wt{\omega} + L_{\wt{Q}} \wt{\beta} ,
        \end{align}
        where $\wt{\beta}$ is given by
        \begin{multline}
        \label{eq:beta}
            \wt{\beta}(\wt{x}, \wt{y}) = (-1)^{k+1} \Big( \bigp{
            \wt{\omega} (\wt{H} \wt{x}, \wt{y}) + 
            (-1)^{|\wt{x}|+1} \wt{\omega} (\wt{x}, \wt{H} \wt{y})} \\
            + \frac{1}{2}\bigp{-\wt{\omega}(\wt{H} \wt{x}, \wt{H} \wt{\dd}_Q \wt{y}) + 
            (-1)^{|\wt{x}} \wt{\omega}(\wt{H}\wt{\dd}_Q \wt{x}, \wt{H} \wt{y}} - 
            \wt{\omega}(\wt{H}\wt{x}, \wt{\dd}_Q \wt{H} \wt{y})\Big).
        \end{multline}
    \end{enumerate}
\end{theorem}

\begin{figure}
    \centering
    \begin{tikzcd}
    \ldots \arrow[r, "\dd_Q"] &
    \FM^{\, n-1} \arrow[r, "\dd_Q"] \arrow[d, xshift=1mm, "i"] &
    \FM^{\, n} \arrow[r, "\dd_Q"] \arrow[d, xshift=1mm,"i"] &
    \FM^{\, n+1} \arrow[r, "\dd_Q"] \arrow[d, xshift=1mm,"i"] & \ldots \\
    \ldots \arrow[r, "\widetilde{\dd}_{Q}"] &
    \FMS^{\, n-1} \arrow[r, "\widetilde{\dd}_{Q}"] \arrow[u, xshift=-1mm, "p"] &
    \FMS^{\, n} \arrow[r, "\widetilde{\dd}_{Q}"] \arrow[u, xshift=-1mm, "p"] &
    \FMS^{\, n+1} \arrow[r, "\widetilde{\dd}_{Q}"] \arrow[u, xshift=-1mm, "p"] & \ldots
\end{tikzcd}
    \caption{Diagram illustrating the requirement of \Cref{enum:theor_a} of \Cref{theor:gluing}. Notation $\FM^{\,n}$ denotes the space of fields in $\FM$ with ghost number $n$.}
    \label{fig:chain_map_theorem}
\end{figure}

In the paper \cite{Gluing_BV-BFV}, the authors presented and proved a lemma, which we are not going to prove here, that shows how the two BV-BFV manifolds of \Cref{theor:gluing} are weakly equivalent Poincaré complexes.
In particular, we report the Lemma below for simplicity.

\begin{lemma}
\label{lemma:Poicare}
    Let $(\F, \dd_Q, \omega)$ and $(\wt{\F}, \wt{\dd}_Q, \wt{\omega})$ be two $k$-Poincaré complexes.
    Assume that we have a pair of quasi-isomorphisms $f:\F \rightarrow \wt{\F}$, $g:\wt{\F} \rightarrow \F$ with chain homotopies $H$, $\wt{H}$ and assume that $f^* \wt{\omega} = \omega$.
    Then $(\F, \dd_Q, \omega)$ and $(\wt{\F}, \wt{\dd}_Q, \wt{\omega})$ are weakly equivalent Poincaré complexes with $\beta = 0$ and with $\wt{\beta}$ defined by \Cref{eq:beta}.
\end{lemma}

In our case, the quasi-isomorphisms are $i$ and $p$ (as implied by Equations~\ref{eq:theor_a} and \ref{eq:theo_b}) and \Cref{eq:requ_lemma} ensures that \Cref{lemma:Poicare} can be applied.
Thus, the two BV-BFV manifolds are weakly equivalent.
Furthermore, we notice that \Cref{eq:wt_omega} is held by the first equation of \Cref{enum:theor_a}.

As already mentioned at the beginning of this section, the assumption that $\M$ is closed is not necessary.
It has been done mainly for simplicity and in order to use the language of Poincaré complexes.
But we could allow $\M$ to have a boundary and replace the $(-1)$-symplectic manifold with its relative version, relative to the data on the boundary $\partial \M$ assigned by the theory $\T$.

\subsection{Sketch of Proof of Conjecture~\ref{conj:first}}
\label{subsec:sketch_proof}

Here we report the sketch of proof for Conjecture~\ref{conj:first} proposed by Cattaneo \etal by factoring through cohomology.

Assume that the BV-BFV theory $\T$ is a perturbation of a free theory $\T_0$ for which Conjecture~\ref{conj:second} holds.
In particular, we consider that $\T$ is obtained from $\T_0$ by adding terms of polynomial degree greater than 3.
The cohomological vector field $Q_0 = \dd_Q$ of $\T_0$ is perturbed by terms of polynomial degree greater than 2.
$\F_\M$ is a graded vector space with cohomological vector field $Q_\M$ corresponding to a $L_\infty$ algebra structure on $\F_\M[-1]$.
In this setting, we have a $L_\infty$ map ($i:\F_\M \rightarrow \FMS$) and we want to promote it to a weak equivalence, as described in \Cref{def:weak_1}.
From this point on, we will write $\F_\M \rightarrow \F$ and $\FMS \rightarrow \wt{\F}$ to ease the notation.

\subsubsection{First part}
\label{subsubsec:first_part}

We start by considering the cohomology $\Hom(\F)$\footnote{We use a different font to distinguish the cohomology $\Hom$ from chain homotopies $H$.}
of $\F$ w.r.t. $\dd_Q$ and the cohomology $\Hom(\wt{\F})$ of $\wt{\F}$ w.r.t. $\wt{\dd}_Q$.
We define the contractions $r, j, \wt{r}, \wt{j}$ such that
\begin{equation}
    \begin{tikzcd}  
    (\F, \dd_Q) \arrow[r,yshift=1mm, "r"]
    \arrow[loop left, distance=15, out=192,in=168, "h"] 
    & \Hom(\F) \arrow[l, yshift=-1mm, "j"]
\end{tikzcd},
\qquad
\begin{tikzcd}  
    (\wt{\F}, \wt{\dd}_Q) \arrow[r,yshift=1mm, "\wt{r}"]
    \arrow[loop left, distance=15, out=192,in=168, "\wt{h}"] 
    & \Hom(\wt{\F}) \arrow[l, yshift=-1mm, "\wt{j}"]
\end{tikzcd}
\end{equation}
\begin{equation}
\label{eq:contractions}
    \text{and} \qquad
    i \circ j = \wt{j} \circ i_*, \quad
    i_* \circ r = \wt{r} \circ i, \quad
    i \circ h = \wt{h} \circ i .
\end{equation}
In the equation above, $i_*$ denotes the push-forward by $i$:
\begin{equation*}
    i_* : \Hom(\F) \rightarrow \Hom(\wt{\F}) ,
\end{equation*}
which is a linear isomorphism in cohomology.

\begin{figure}
    \centering
    \begin{tikzcd}
    & \F \arrow[r, yshift=1mm, "r"]
    \arrow[d, swap, "i"]
    \arrow[loop left, distance=13, out=160, in=110, "h"]
    & \Hom(\F) \arrow[l, yshift=-1mm, "j"]
    \arrow[d, "i_*"] \\
    & \wt{\F} \arrow[r, yshift=1mm, "\wt{r}"]
    \arrow[loop left, distance=13, out=-120, in=184, "\wt{h}"]
    & \Hom(\wt{\F}) \arrow[l, yshift=-1mm, "\wt{j}"]
\end{tikzcd}
    \caption{Commutative diagram displaying the contractions maps in \Cref{eq:contractions}.}
    \label{fig:contractions_diagram}
\end{figure}

In the paper, Cattaneo \etal use the homological perturbation lemma (see \cite[Section 6.4]{Hom_mirror_trasnfer_lemma}) in order to deform the contraction maps $r, j, \wt{r}, \wt{j}$ into the $L_\infty$ maps $\mathpzc{j}, \mathpzc{r}, \wt{\mathpzc{j}}, \wt{\mathpzc{r}}$, such that
\begin{equation}
    \begin{tikzcd}  
    (\F, Q) \arrow[r,yshift=1mm, "\cazz{r}"]
    \arrow[loop left, distance=15, out=192,in=168, "\mathcal{H}"] 
    & (\Hom(\F), Q_\Hom) \arrow[l, yshift=-1mm, "\cazz{j}"]
\end{tikzcd},
\qquad
\begin{tikzcd}  
    (\wt{\F}, \wt{Q}) \arrow[r,yshift=1mm, "\wt{\cazz{r}}"]
    \arrow[loop left, distance=15, out=192,in=168, "\wt{\mathcal{H}}"] 
    & (\Hom(\wt{\F}), \wt{Q}_\Hom) \arrow[l, yshift=-1mm, "\wt{\cazz{j}}"]
\end{tikzcd} .
\end{equation}
Here, $\mathcal{H}$, $\wt{\mathcal{H}}$ denote the $L_\infty$ homotopies and $Q_\Hom$, $\wt{Q}_\Hom$ the induced minimal $L_\infty$ algebra structures on cohomology.
With these new $L_\infty$ maps, Equations~\ref{eq:contractions} become
\begin{equation}
    i \circ \cazz{j} = \wt{\cazz{j}} \circ i_*, \quad
    i_* \circ \cazz{r} = \wt{\cazz{r}} \circ i, \quad
    i \circ \mc{H} = \wt{\mc{H}} \circ i .
\end{equation}

Now, in order to lead back to \Cref{eq:theorem_maps}, we need to define the map $p$\footnote{Step taken from \cite[Section 10.4.6]{Algebraic_operands_key_step}, as the authors write.}.
We construct it as
\begin{equation}
\label{eq:def_p}
    p = \cazz{j} \circ (i_*)^{-1} \circ \wt{\cazz{r}}:
    \; \wt{\F} \rightarrow \F .
\end{equation}
\begin{figure}
    \centering
    \begin{tikzcd}
    & \F
    & \Hom(\F) \arrow[l, swap, "\cazz{j}"] \\
    & \wt{\F} \arrow[u, line width=0.3mm, "p"] \arrow[r, "\wt{\cazz{r}}"]
    & \Hom(\wt{\F} \arrow[u, swap, "(i_*)^{-1}"]
\end{tikzcd}
    \caption{Diagram displaying definition of $p$ as in \Cref{eq:def_p}.}
    \label{fig:comm_diagram_def_p}
\end{figure}
In this way, $p$ is a quasi-inverse of the canonical inclusion $i$.
We get the diagram of maps
\begin{equation}
    \begin{tikzcd}  
    (\F, Q) \arrow[r,yshift=1mm, "i"]
    \arrow[loop left, distance=20, out=198,in=162, "\mathcal{H}"] 
    & (\wt{\F}, \wt{Q}) \arrow[l, yshift=-1mm, "p"]
    \arrow[loop right, distance=20,out=18,in=-18, "\wt{\mc{H}}"]
\end{tikzcd},
\end{equation}
where $i$ and $p$ are quasi-inverse $L_\infty$ quasi-isomorphisms and the $L_\infty$ homotopies $\mc{H}$, $\wt{\mc{H}}$ satisfy
\begin{equation}
\label{eq:L_infty_homotopies}
    \begin{aligned}
        Q \mc{H} + \mc{H} Q =&
        \; \id - p \circ i , \\
        \wt{Q} \wt{\mc{H}} + \wt{\mc{H}} \wt{Q} =&
        \; \id - i \circ p .
    \end{aligned}
\end{equation}

Thus, $i$ and $p$ are the pair of dg maps between $\F$ and $\wt{\F}$ of \Cref{def:weak_1}.
In particular, their pullbacks are quasi-isomorphisms of chain complexes:
\begin{equation}
    \begin{tikzcd}  
    (C^\infty(\F), Q) \arrow[r,yshift=1mm, "i^*"]
    \arrow[loop left, distance=20, out=192,in=168, "\mathcal{H}"] 
    & (C^\infty(\wt{\F}), \wt{Q}) \arrow[l, yshift=-1mm, "p^*"]
    \arrow[loop right, distance=20,out=12,in=-12, "\wt{\mc{H}}"]
\end{tikzcd} .
\end{equation}
This is \Cref{enum:weak_a} of the definition of weak equivalence between $k$-symplectic manifolds.

\subsubsection{Second part}
\label{subsubsec:second_part}

As for \Cref{enum:weak_b}, we repeat the construction above, but applied to the tangent lifts from $\F$, $\wt{\F}$ to the shifted tangent bundles $T[1]\F$, $T[1]\wt{\F}$.
In particular, we doubled the constractions in the first Equation~\ref{eq:contractions} to
\begin{equation}
    \begin{tikzcd}  
    (T[1]\F, \dd_Q \oplus \dd_Q) \arrow[r,yshift=1mm, "r \oplus r"]
    \arrow[loop left, distance=15, out=188,in=172, "h \oplus h"] 
    & T[1]\Hom(\F) \arrow[l, yshift=-1mm, "j \oplus j"]
\end{tikzcd},
\end{equation}
and similarly for the second equation with $\wt{\F}$.
We think of $T[1]\F$ as $\F \oplus \F[1]$ with an extra differential $\delta$ mapping between $\F$ and $\F[1]$.
Then we use the homological perturbation lemma, yielding to
\begin{equation}
    \begin{tikzcd}  
    (\Omega^\bullet(\F), L_Q) \arrow[r,yshift=1mm, "\cazz{r}^*"]
    \arrow[loop left, distance=15, out=190, in=170, "\mc{H}^{\text{lifted}}"] 
    & (\Omega^\bullet(\Hom(\F)), L_{Q_\Hom}) \arrow[l, yshift=-1mm, "\cazz{j}^*"]
\end{tikzcd}
\end{equation}
and again similar for the tilde-side.

Then we repeat Construction~\ref{eq:def_p} to define the map $p^*$, quasi-inverse of $i^*$, but at the level of $T[1]\F$ and $t[1]\wt{\F}$.
We obtain the chain equivalence
\begin{equation}
\label{eq:maps_sketch_proof_part_b}
    \begin{tikzcd}  
    (\Omega^\bullet(\F), L_Q) \arrow[r,yshift=1mm, "i^*"]
    \arrow[loop left, distance=20, out=192,in=168, "\mathcal{H}^\text{lifted}"] 
    & (\Omega^\bullet (\wt{\F}), L_{\wt{Q}}) \arrow[l, yshift=-1mm, "p^*"]
    \arrow[loop right, distance=20,out=12,in=-12, "\wt{\mc{H}}^\text{lifted}"]
\end{tikzcd} .
\end{equation}

By fiber product construction we have 
\begin{equation*}
    i^* \wt{\omega} = \omega .
\end{equation*}
We can use this to obtain
\begin{equation*}
    p^* \omega =
    p^* i^* \wt{\omega}
    \stackrel{\ref{eq:L_infty_homotopies}}{=}
    (\id - L_{\wt{Q}} \wt{\mc{H}}^\text{lifted} - \wt{\mc{H}}^\text{lifted} L_{\wt{Q}}) \wt{\omega} =
    \wt{\omega} + L_{\wt{Q}} (- \wt{\mc{H}}^\text{lifted} \wt{\omega}) =
    \wt{\omega} + L_{\wt{Q}} \wt{\beta},
\end{equation*}
where we used $L_{\wt{Q}} \wt{\omega}=0$ and $\wt{\beta}=- \wt{\mc{H}}^\text{lifted} \wt{\omega}$.

Note that, since $\delta$ commutes with $\wt{\mc{H}}^\text{lifted}$ and with all the other maps of \Cref{eq:maps_sketch_proof_part_b}, $\wt{\beta}$ is a closed 2-form w.r.t. the differential $\delta$, in fact
\begin{equation*}
    \delta \wt{\beta} = 
    - \delta \wt{\mc{H}}^\text{lifted} \wt{\omega} =
    - \wt{\mc{H}}^\text{lifted} \delta \wt{\omega} =
    0 .
\end{equation*}

Thus, we have recovered \Cref{enum:weak_b} of \Cref{def:weak_1} with $\beta = 0$ and $\wt{\beta}=- \wt{\mc{H}}^\text{lifted} \wt{\omega}$.
