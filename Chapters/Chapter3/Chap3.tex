\chapter{The BV-BFV formalism}
\label{chap:BV-BFV}

In this chapter we will first talk about the need of a new formalism in order to include \emph{gauge theories} when computing Feynman diagrams.
We will then present the cure of this problem, which is the Batalin-Vilkovisky construction.
Finally, we will introduce the definition of a manifold with boundary in the BV-BFV formalism.

The whole chapter is mainly based on \cite{Nima}, but it also has some references to \cite{Musio},\cite{Mnfd_boundaries} and \cite{Lectures}.

\section{Motivation}
\label{sec:motivation}

In the path integral approach, the partition function is constructed as 
\begin{equation}
\label{eq:parti_func}
    \mathcal{Z}_M = \int_{F_M} e^{\frac{i}{\hbar}S_M(\phi)} \mathcal{D}\phi .
\end{equation}
Here $F_M$ is the space of fields, with $\phi \in F_M$, and $S_M$ denotes the action functional of the theory.
$\mathcal{D} \phi$ is the measure on the space of fields $F_M$, but this is generally problematic since the latter can be infinite dimensional.
Therefore the measure $\mathcal{D} \phi$ is usually defined as an asymptotic series in $\hbar \rightarrow 0$.

The main idea behind this is that the fast oscillations of the terms cancel out in the integral, except for the fields near the critical points of the action functional $S_M$.
By expanding the exponential, one is able to write \Cref{eq:parti_func} as a series of Feynman diagrams.

The main problem with this perturbative approach is that it requires the critical points of the action functional to be isolated (more precisely, it requires the Hessian matrix at the critical points to be non-degenerate).
This does not allow us to consider gauge theories in the solutions.

Gauge theories shows up as a symmetry of the Lagrangian function under certain local transformations.
Thus, there must be an orbit in the space of fields $F_M$, to which our solution belongs, that also preserves the action functional $S_M$.
In other words, the Hessian matrix is degenerate in the direction of the orbit.

The way out from this problem is through the BV formalism.
We want to replace the classical data $(F_M, S_M)$ with a triple $(\mathcal{F}_M, \omega_M, \mathcal{S}_M)$, in order to rewrite the partition function in  \Cref{eq:parti_func} as
\begin{equation}
\label{eq:part_func_M}
    \mathcal{Z}_M = \int_{\mathcal{L} \subset \mathcal{F}_M} e^{\frac{i}{\hbar} \mathcal{S}_M} \mathcal{D} \phi
\end{equation}
where $\mathcal{L}$ is a Lagrangian submanifold w.r.t. the symplectic structure $\omega_M$.
This time, in case of gauge theories, \Cref{eq:part_func_M} is well-defined and is invariant under deformations of $\mathcal{L}$.
\section{The Batalin–Vilkovisky Formalism}
\label{sec:BV_formalism}

\begin{definition}
\label{def:BV_manifold}
    A \emph{BV manifold} is a triple $(\mathcal{F}_M, \omega_M, \mathcal{S}_M)$ where
    \begin{itemize}
        \item $\mathcal{F}_M$ is a supermanifold with an additional $\mathbb{Z}$-grading;
        \item $\omega_M$ an odd symplectic form of degree $-1$;
        \item $\mathcal{S}_M$ an even function on $\mathcal{F}_M$ of degree $0$, satisfying the \emph{Classical Master Equation} (CME)
        \begin{equation}
            \{\mathcal{S}_M, \mathcal{S}_M\} = 0
        \end{equation}
        where $\{\;,\,\}$ denotes the Poisson bracket induces by $\omega_M$.
    \end{itemize}
\end{definition}

$\mathcal{F}_M$ is usually called the \emph{space of fields} and the function $\mathcal{S}_M$ is the \emph{action functional}.

Similarly to what we did for symplectic manifolds in \Cref{sec:symplectic_geom}, we can consider the Hamiltonian vector field $Q$ given by the Hamiltonian function $\mathcal{S}_M$ with respect to the symplectic form $\omega_M$, defined by the equation
\begin{equation*}
    \iota_Q \omega_M = \dd \mathcal{S}_M.
\end{equation*}

Using Cartan's magic formula:
\begin{align*}
    L_Q \omega_M = \iota_Q \dd \omega_M + \dd \iota_Q \omega_M =
    0 + \dd^2 \mathcal{S}_M = 0
\end{align*}
since $\omega_M$ is closed for being a symplectic form.
This implies that the vector field $Q$ is indeed \emph{symplectic}.

Furthermore, the vector field $Q$ is also \emph{cohomological}, i.e. $Q^2 = 0$.
Since we have chosen $\omega_M$ to be of degree $-1$, this implies that $Q$ is of degree $+1$ and thus defines a de Rham differential.
Therefore, it endows $C^\infty(\mathcal{F}_M)$ with the structure of a cochain complex.

Notice that is possible to define the vector field $Q$ also through
\begin{equation*}
    Q = \{\mathcal{S}_M, \;\; \}.
\end{equation*}
It is therefore possible to write the Classical Master Equation as
\begin{equation}
\label{eq:QS0}
    Q(\mathcal{S}_M) = 0 .
\end{equation}
Since $Q$ plays the role of a differential in this formalism, \Cref{eq:QS0} is the equivalent of $\delta S = 0$ in classical mechanics, which then leads us to the Euler-Lagrange equation
\begin{equation}
    \dv{t} \pdv{L}{\dot{q}} - \pdv{L}{q} = 0 .
\end{equation}
\section{The BV-BFV formalism}
\label{sec:BF-BFV}

The BV-BFV formalism was introduced in order to define manifolds with boundaries inside the BV formalism.
For this purpose, we need to define a new type of manifold which will be the boundary of our BV manifold.

\begin{definition}
\label{def:BFV_manifold}
    A \emph{BFV manifold} is a triple $(\mathcal{F}_\Sigma, \omega_\Sigma, \mathcal{S}_\Sigma)$ where
    \begin{itemize}
        \item $\mathcal{F}_\Sigma$ is a graded supermanifold;
        \item $\omega_\Sigma$ an even symplectic form of degree $0$;
        \item $\mathcal{S}_\Sigma$ an odd function on $\mathcal{F}_\Sigma$ of degree $+1$, satisfying the Classical Master Equation $\{\mathcal{S}_\Sigma, \mathcal{S}_\Sigma\} = 0$.
    \end{itemize}
    The BFV manifold is called \emph{exact} if there exists a $1$-form $\alpha_\Sigma$ such that $\omega_\Sigma = \delta \alpha_\Sigma$.
\end{definition}

In the definition above, $\delta$ denotes the de Rham differential, generally given by the cohomological vector field $Q_\Sigma = \{\mathcal{S}_\Sigma, \;\;\}$.

%\newpage
\begin{definition}
\label{def:BV-BFV}
    A \emph{BV-BFV manifold} over a BFV manifold $(\mathcal{F}_\Sigma, \omega_\Sigma = \delta \alpha_\Sigma, \mathcal{S}_\Sigma)$ is a quintuple $(\mathcal{F}_M, \omega_M, \mathcal{S}_M, Q_M, \pi_{M, \Sigma})$ where
    \begin{itemize}
        \item $\mathcal{F}_M$ is a graded supermanifold;
        \item $\omega_M$ is an odd symplectic form of degree $-1$;
        \item $\mathcal{S}_M$ is an even function of degree $0$;
        \item $Q_M$ is a cohomological vector field of degree $+1$;
        \item $\pi_{M, \Sigma}: \mathcal{F}_M \rightarrow \mathcal{F}_\Sigma$ is a surjective submersion satisfying
        \begin{equation}
            \iota_{Q_M} \omega_M = \delta \mathcal{S}_M - \pi^*_{M,\Sigma} \alpha_\Sigma.
        \end{equation}
    \end{itemize}
\end{definition}

Notice that the purpose of the map $\pi_{M, \Sigma}$ in the above definition is to link the fields on the bulk $M$ to the boundary fields on $\Sigma$.
With $\pi_{M,\Sigma}^*$ we denote the pullback from the space of $1$-forms on $\mathcal{F}_\Sigma$ to the one on $\mathcal{F}_M$, given by $\pi_{M,\Sigma}^* \alpha_\Sigma = \alpha_\Sigma \circ \pi_{M,\Sigma}$.

In the definition above, the BV-BFV manifold $(\mathcal{F}_M, \omega_M, \mathcal{S}_M, Q_M, \pi_{M, \Sigma})$ is often called \emph{relative} to the BFV manifold $(\mathcal{F}_\Sigma, \omega_\Sigma, \mathcal{S}_\Sigma)$.

On the paper \cite{mCME}, the authors add also the condition $\delta \pi Q = Q_\Sigma$ to \Cref{def:BV-BFV}, where $\delta \pi$ denotes the differential of $\pi$ and $Q_\Sigma$ the cohomological vector field of the function $\mathcal{S}_\Sigma$.
This leads to the \emph{modified Classic Master Equation} (mCME)
\begin{equation}
\label{eq:mCME}
    Q_M (\mathcal{S}_M) = \pi ^* (2 \mathcal{S}_\Sigma - \iota_{Q_\Sigma}(\alpha_\Sigma)).
\end{equation}
If $\mathcal{F}_\Sigma$ is a point, then the BV-BFV manifold reduces to a BV manifold and the mCME reduces to the usual CME $\{\mathcal{S}_M, \mathcal{S}_M\} = 0$.

\Cref{eq:mCME} can be equivalently written as
\begin{equation*}
    \frac{1}{2} \iota_{Q_M} \iota_{Q_M} \omega_M =  \pi^* \mathcal{S}_\Sigma .
\end{equation*}

More generally, BV manifolds as in \Cref{def:BV_manifold} are also called $(-1)$-Hamiltonian manifolds and BFV manifolds as in \Cref{def:BFV_manifold} are known as $0$-Hamiltonian manifolds.
In particular, the numbers $-1$ and $0$ in front of the Hamiltonian manifolds stand for the degree of the symplectic form.
In this way, it is possible to deduce the definition of $k$-Hamiltonian manifolds from the one of BV manifolds by generalising the degree of the symplectic form.

If a $k$-Hamiltonian manifold is not equipped with its function $\mathcal{S}$, then the manifold is called a $k$-symplectic manifold, generalizing the concept of symplectic manifolds we introduced in \Cref{def:sympl_mnf} to the case of graded supermanifolds.

In the next chapters, we will often refer to this kind of manifolds which are differential and graded as differential graded manifolds, or just with the abbreviation “dg”.