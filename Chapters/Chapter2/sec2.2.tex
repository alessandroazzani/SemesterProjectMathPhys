\section{Supermanifolds and Graded Manifolds}
\label{sec:supermanifolds}

Similarly to the linear case, we now generalize the concepts of superspace and graded vector space to the case of manifolds.
We start by the definition of supermanifolds.

\begin{definition}
\label{def:supermanifold}
    A \emph{supermanifold} $\mathcal{M}$ is a locally ringed space $(M, \mathcal{O}_M)$, which is locally isomorphic to
    \begin{equation*}
        \Big(\mathcal{U}, C^\infty(\mathcal{U}) \otimes \bigwedge V^*\Big),
    \end{equation*}
    where $\mathcal{U}$ is an open subset of $\mathbb{R}^d$ and $V$ some finite-dimensional real vector space.
\end{definition}

We call the manifold $M$ in the previous definition the \emph{body} and $\mathcal{O}_M$ the $\emph{structure sheaf}$ of $M$.
Notice that a supermanifold has locally the same structure as the algebra we described in \Cref{ex:comm_anticomm}.

Notice also that the isomorphism of \Cref{def:supermanifold} is a $\mathbb{Z}_2$-graded algebra, which is defined by the parity operator:
\begin{align*}
    |\quad | : 
    \bigoplus_{k \geq 0} C^\infty(\mathcal{U}) \otimes \bigwedge^k V^* 
    &\rightarrow \mathbb{Z}_2 \\
    f \otimes \theta &
    \mapsto |f \otimes \theta| = |\theta| = k \! \! \mod{2}.
\end{align*}

\begin{example}
    The supermanifold having $\mathbb{R}^q$ as body ($q$ even coordinates) and structure sheaf given by
    \begin{equation*}
         C^\infty(\mathbb{R}^q) \otimes \bigwedge \mathbb{R}^p
    \end{equation*}
    ($p$ odd coordinates) is called the \emph{standard supermanifold} and is denoted by $\mathbb{R}^{q|p}$.
\end{example}

We now introduce another important structure, that can be combined with the one of supermanifolds.

\begin{definition}
    \label{def:graded_manifold}
    A \emph{graded manifold} is a locally ringed space $\mathcal{M} = (M, \mathcal{O}_M)$, which is locally isomorphic to
    \begin{equation*}
        \Big(\mathcal{U}, C^\infty(\mathcal{U}) \otimes \text{Sym}(V^*) \Big),
    \end{equation*}
    where $\mathcal{U}$ is an open subset of $\mathbb{R}^d$ and $V$ some finite-dimensional real vector space.
\end{definition}
 In \Cref{def:graded_manifold}, by $\text{Sym}(V^*)$ we denote the set of symmetric powers of $V^*$, i.e. $\text{Sym}(V^*) \coloneqq \left( \text{Sym}^n(V^*) \right)_{n \in \mathbb{N}}$. The $n$-th symmetric power of a vector space $V$ is given by
 \begin{equation*}
     \text{Sym}^n V =
     V^{\otimes n} /
     \left( x_1 \otimes \ldots \otimes x_n - \sigma (x_1 \otimes \ldots \otimes x_n,
     \sigma \in \mathfrak{S}_n \right),
 \end{equation*}
 where $\mathfrak{S}_n$ denotes the $n$-th symmetric group.

The $\mathbb{Z}$-grading of a graded manifold is denoted by “gh" and called the \emph{ghost number}.
In particular, for some function $f$ on a graded manfiold $\mathcal{M}$, we denote by $\text{gh}(f)$ its degree with respect to the $\mathbb{Z}$-grading.
It is assumed that physical fields have ghost number zero.
This construction is useful since, in field theory, the space of fields can be described with the structure of a graded manifold.

Instead, the $\mathbb{Z}_2$-grading of a supermanifold corresponds to the commuting and anti-commuting coordinates.
In physics language, the property of commuting even coordinates corresponds to bosonic particles, while the anti-commuting property of the odd coordinates is specific for fermionic particles.