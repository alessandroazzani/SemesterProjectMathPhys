\section{Supergeometry in the Linear case}
\label{sec:Linear_case}

Before looking at the notion of \emph{supermanifolds} and \emph{graded manifolds}, we introduce these concepts in the linear case.
Therefore, in this section we will define graded vector spaces and superspaces, as well as some of their properties, to get us familiar with these structures.
Some of these definition were taken from \cite{Binda}.

\begin{definition}
    A \emph{graded vector space $V$} is a collection of vector spaces $(V_k)_{k \in \mathcal{I}}$, where $\mathcal{I}$ is an index set.
    When the index set $\mathcal{I}$ is $\mathbb{Z}$, it is generally called \emph{$\mathbb{Z}$-graded vector space}.
\end{definition}

From now on, when we talk about graded vector spaces, we will be referring to $\mathbb{Z}$-graded vector spaces.
Some authors prefer to talk about direct sums of vector spaces instead of a collection.
We rather avoid this definition because it will be easier to talk about morphisms between the spaces.

Furthermore, given an element $v \in V_k$, $k$ is called the \emph{degree} of $v$.

\begin{definition}
    Let $V = (V_k)_{k \in \mathbb{Z}}$ be a graded vector space.
    We can define the \emph{$n$-shift} of $V$ by an integer $n$ by
    \begin{equation*}
        V[n] \coloneqq (V_{k + n})_{k \in \mathbb{Z}} .
    \end{equation*}
\end{definition}

\begin{definition}
    A \emph{superspace} is a $\mathbb{Z}_2$-graded vector space $V = V_{\bar{0}} \oplus V_{\bar{1}}$.
    We call $V_{\bar{0}}$ the vector space of \emph{even} vectors and $V_{\bar{1}}$ the space of \emph{odd} vectors.
\end{definition}

Notice that a superspace is a graded vector space but with a different grading.
Together with the concept of superspace comes the parity of it.

\begin{definition}
    The \emph{parity}, or \emph{degree}, of a superspace $V$ is its $\mathbb{Z}_2$ grading.
    The degree of an element $v \in V$ is denoted by $|v|$ and given by
    \begin{equation*}
        |v| \coloneqq
        \begin{cases}
            \text{even}, \quad v \in V_{\bar{0}} \\
            \text{odd}, \quad v \in V_{\bar{1}}
        \end{cases}.
    \end{equation*}
\end{definition}

If we consider the superspace $V$ to be a regular vector space and an automophism $P: V \rightarrow V$ such that $P^2 = \text{id}$, then we get that an eigenvector $v \in V$ has a $+1$ eigenvalue if it has even parity or, vice versa, if it has an odd parity then its eigenvalue is $-1$.
Moreover, we can write $P v = (-1)^{|v|}v$.

Notice that every graded vector space $V$ can be regarded as a superspace by setting
\begin{align*}
    V_{\bar{0}} &= \bigoplus_{k \in 2 \mathbb{Z}} V_k \\
    V_{\bar{1}} &= \bigoplus_{k \in 2 \mathbb{Z} + 1} V_k .
\end{align*}
Thus, in this case the parity is given by the degree of the subspace modulo 2.

Given a superspace $V$ and the automorphism $P$, one can define the \emph{change of parity}
\begin{equation*}
    \Pi : (V, P) \mapsto (V, -P) .
\end{equation*}
In this way, we get
\begin{align*}
    (\Pi V)_{\bar{0}} &= V_{\bar{1}} \\
    (\Pi V)_{\bar{1}} &= V_{\bar{0}} .
\end{align*}

In the case of a graded vector space, the change of parity is defined as the $1$-shift of $V$, thus $\Pi V= V[1]$.

\begin{example}
\label{ex:comm_anticomm}
    Locally, we can consider even coordinates ($x^i$) on an open subset $\mathcal{U}$ of $\mathbb{R}$ and the algebra of smooth function $C^\infty (\mathcal{U})$, which are described by their commutativity, i.e. $x^i x^j = x^j x^i$.
    We can add anticommutating coordinates ($\theta^\mu$) such that, algebraically, we obtain
    \begin{align*}
        \theta^\mu x^i &= x^i \theta^\mu \\
        \theta^\mu \theta^\nu &= \theta^\nu \theta^\mu.
    \end{align*}
    We can write the algebra generates by these coordinates as
    \begin{equation*}
        \widehat{\mathbb{R}[x, \theta]} = C^\infty (\mathcal{U}) \otimes \bigwedge V^*
    \end{equation*}
    for some vector space $V$.
\end{example}