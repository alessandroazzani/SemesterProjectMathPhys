\chapter[Introduction to Supergeometry]{Introduction to \\ Supergeometry}
\label{chap:intro_supergeom}

Supergeometry is an essential concept in order to understand the Batalin-Vilkovisky formalism and its adaptation to the Batalin-Fradkin-Vilkovisky formalism.

In this chapter, we start with the easier concepts of \emph{graded vector fields} and \emph{superspaces}, before adapting these to the case of manifolds.
We define \emph{supermanifolds} and \emph{graded manifolds} and give some examples.
Finally, we see how to build vector fields on these complicated structures.

Most of the content of this chapter is taken from \cite{Nima} and \cite{Intro_BV-BFV}.


\section{Supergeometry in the Linear case}
\label{sec:Linear_case}

Before looking at the notion of \emph{supermanifolds} and \emph{graded manifolds}, we introduce these concepts in the linear case.
Therefore, in this section we will define graded vector spaces and superspaces, as well as some of their properties, to get us familiar with these structures.
Some of these definition were taken from \cite{Binda}.

\begin{definition}
    A \emph{graded vector space $V$} is a collection of vector spaces $(V_k)_{k \in \mathcal{I}}$, where $\mathcal{I}$ is an index set.
    When the index set $\mathcal{I}$ is $\mathbb{Z}$, it is generally called \emph{$\mathbb{Z}$-graded vector space}.
\end{definition}

From now on, when we talk about graded vector spaces, we will be referring to $\mathbb{Z}$-graded vector spaces.
Some authors prefer to talk about direct sums of vector spaces instead of a collection.
We rather avoid this definition because it will be easier to talk about morphisms between the spaces.

Furthermore, given an element $v \in V_k$, $k$ is called the \emph{degree} of $v$.

\begin{definition}
    Let $V = (V_k)_{k \in \mathbb{Z}}$ be a graded vector space.
    We can define the \emph{$n$-shift} of $V$ by an integer $n$ by
    \begin{equation*}
        V[n] \coloneqq (V_{k + n})_{k \in \mathbb{Z}} .
    \end{equation*}
\end{definition}

\begin{definition}
    A \emph{superspace} is a $\mathbb{Z}_2$-graded vector space $V = V_{\bar{0}} \oplus V_{\bar{1}}$.
    We call $V_{\bar{0}}$ the vector space of \emph{even} vectors and $V_{\bar{1}}$ the space of \emph{odd} vectors.
\end{definition}

Notice that a superspace is a graded vector space but with a different grading.
Together with the concept of superspace comes the parity of it.

\begin{definition}
    The \emph{parity}, or \emph{degree}, of a superspace $V$ is its $\mathbb{Z}_2$ grading.
    The degree of an element $v \in V$ is denoted by $|v|$ and given by
    \begin{equation*}
        |v| \coloneqq
        \begin{cases}
            \text{even}, \quad v \in V_{\bar{0}} \\
            \text{odd}, \quad v \in V_{\bar{1}}
        \end{cases}.
    \end{equation*}
\end{definition}

If we consider the superspace $V$ to be a regular vector space and an automophism $P: V \rightarrow V$ such that $P^2 = \text{id}$, then we get that an eigenvector $v \in V$ has a $+1$ eigenvalue if it has even parity or, vice versa, if it has an odd parity then its eigenvalue is $-1$.
Moreover, we can write $P v = (-1)^{|v|}v$.

Notice that every graded vector space $V$ can be regarded as a superspace by setting
\begin{align*}
    V_{\bar{0}} &= \bigoplus_{k \in 2 \mathbb{Z}} V_k \\
    V_{\bar{1}} &= \bigoplus_{k \in 2 \mathbb{Z} + 1} V_k .
\end{align*}
Thus, in this case the parity is given by the degree of the subspace modulo 2.

Given a superspace $V$ and the automorphism $P$, one can define the \emph{change of parity}
\begin{equation*}
    \Pi : (V, P) \mapsto (V, -P) .
\end{equation*}
In this way, we get
\begin{align*}
    (\Pi V)_{\bar{0}} &= V_{\bar{1}} \\
    (\Pi V)_{\bar{1}} &= V_{\bar{0}} .
\end{align*}

In the case of a graded vector space, the change of parity is defined as the $1$-shift of $V$, thus $\Pi V= V[1]$.

\begin{example}
\label{ex:comm_anticomm}
    Locally, we can consider even coordinates ($x^i$) on an open subset $\mathcal{U}$ of $\mathbb{R}$ and the algebra of smooth function $C^\infty (\mathcal{U})$, which are described by their commutativity, i.e. $x^i x^j = x^j x^i$.
    We can add anticommutating coordinates ($\theta^\mu$) such that, algebraically, we obtain
    \begin{align*}
        \theta^\mu x^i &= x^i \theta^\mu \\
        \theta^\mu \theta^\nu &= \theta^\nu \theta^\mu.
    \end{align*}
    We can write the algebra generates by these coordinates as
    \begin{equation*}
        \widehat{\mathbb{R}[x, \theta]} = C^\infty (\mathcal{U}) \otimes \bigwedge V^*
    \end{equation*}
    for some vector space $V$.
\end{example}
\section{Supermanifolds and Graded Manifolds}
\label{sec:supermanifolds}

Similarly to the linear case, we want to generalize the concepts of superspace and graded vector space to the case of manifolds.
For this we start by the definition of supermanifolds.

\begin{definition}
\label{def:supermanifold}
    A \emph{supermanifold} $\mathcal{M}$ is a locally ringed space $(M, \mathcal{O}_M)$, which is locally isomorphic to
    \begin{equation*}
        \Big(\mathcal{U}, C^\infty(\mathcal{U}) \otimes \bigwedge V^*\Big),
    \end{equation*}
    where $\mathcal{U}$ is an open subset of $\mathbb{R}^d$ and $V$ some finite-dimensional real vector space.
\end{definition}

We call the manifold $M$ in the previous definition the \emph{body} and $\mathcal{O}_M$ the $\emph{structure sheaf}$ of $M$.
Notice that a supermanifold has locally the same structure as the algebra we described in \Cref{ex:comm_anticomm}.

Notice also that the isomorphism present in \Cref{def:supermanifold} is a $\mathbb{Z}_2$-graded algebra, which is defined by the parity operator:
\begin{align*}
    |\quad | : 
    \bigoplus_{k \geq 0} C^\infty(\mathcal{U}) \otimes \bigwedge^k V^* 
    &\rightarrow \mathbb{Z}_2 \\
    f \otimes \theta &
    \mapsto |f \otimes \theta| = |\theta| = k \! \! \mod{2}.
\end{align*}

\begin{example}
    The supermanifold having $\mathbb{R}^q$ as body ($q$ even coordinates) and structure sheaf given by
    \begin{equation*}
         C^\infty(\mathbb{R}^q) \otimes \bigwedge \mathbb{R}^p
    \end{equation*}
    ($p$ odd coordinates) is called the \emph{standard supermanifold} and denoted by $\mathbb{R}^{q|p}$.
\end{example}

We now introduce another important structure, that can be combined with the one of supermanifolds.

\begin{definition}
    \label{def:graded_manifold}
    A \emph{graded manifold} is a locally ringed space $\mathcal{M} = (M, \mathcal{O}_M)$, which is locally isomorphic to
    \begin{equation*}
        \Big(\mathcal{U}, C^\infty(\mathcal{U}) \otimes \text{Sym}(V^*) \Big),
    \end{equation*}
    where $\mathcal{U}$ is an open subset of $\mathbb{R}^d$ and $V$ some finite-dimensional real vector space.
\end{definition}
 In \Cref{def:graded_manifold}, by $\text{Sym}(V^*)$ we denote the set of symmetric powers of $V^*$, i.e. $\text{Sym}(V^*) \coloneqq \left( \text{Sym}^n(V^*) \right)_{n \in \mathbb{N}}$. The $n$-th symmetric power of a vector space $V$ is given by
 \begin{equation*}
     \text{Sym}^n V =
     V^{\otimes n} /
     \left( x_1 \otimes \ldots \otimes x_n - \sigma (x_1 \otimes \ldots \otimes x_n,
     \sigma \in \mathfrak{S}_n \right),
 \end{equation*}
 where $\mathfrak{S}_n$ denotes the $n$-th symmetric group.

The $\mathbb{Z}$-grading of a graded manifold is denoted by gh and named \emph{ghost number}.
In particular, for some function $f$ on a graded manfiold $\mathcal{M}$, we denote by $\text{gh}(f)$ its degree with respect to the $\mathbb{Z}$-grading.
In particular, physical fields have ghost number zero.
This is useful since in field theory, the space of fields can be describes with the structure of a graded manifold.

Instead, the $\mathbb{Z}_2$-grading of a supermanifold corresponds to the commuting and anticommuting coordinates.
In physics language, the property of commuting even coordinates corresponds to bosonic particles, while the anticommuting property of the odd coordinates is specific for fermionic particles.
\section{Super and Graded Vector Fields}
\label{sec:Sup_vector_fields}

In this section we will introduce the last concepts in order to understand the notion of BV-BFV theory.
We will briefly see how to define vector fields on supermanifold and graded manifolds.
We will then make some remarks on the concept of differential forms and finally introduce the \emph{cohomological vector field}.

\begin{definition}
    A \emph{supervector field} $X$ is a vector field on a supermanifold $\mathcal{M}$, which in local coordinates $(x^i, \theta^\mu)$ can be written as
    \begin{equation*}
        X = \sum_{i, \mu} X^i \partial_{x^i} + X^\mu \partial_{\theta^\mu} .
    \end{equation*}
\end{definition}

\begin{definition}
    A \emph{graded vector field} $X$ of degree $k$ on a graded manifold $\mathcal{M}$ is a graded linear map $X: C^\infty(\mathcal{M}) \rightarrow C^\infty(\mathcal{M})[k]$, which satisfies the Liebniz rule
    \begin{equation*}
        X(fg) = X(f)g + (-1)^{k|f|} f X(g)
    \end{equation*}
    for two functions $f,g \in C^\infty(\mathcal{M})$.
\end{definition}

We will use a $(+1)$-vector field $Q: C^\infty(\mathcal{M}) \rightarrow C^\infty(\mathcal{M})[1]$ as a differential on the manifold $\mathcal{M}$.
Furthermore, we will require the vector field $Q$ to be \emph{cohomological}, i.e. $Q^2 = 0$.
In particular, this will be used later on in \Cref{chap:gluing} as our de Rham differential.

\begin{definition}
    A \emph{graded symplectic form} $\omega$ of degree $k$ on a graded manifold $\mathcal{M}$ is a closed, non-degenerate $2$-form
    \begin{equation*}
        \omega : T\mathcal{M} \rightarrow T^*[k]\mathcal{M}
    \end{equation*}
    whch can be written in local coordinates as
    \begin{equation*}
        \omega = \sum_{i,j} dy^i \omega_{ij} dy^j
    \end{equation*}
    with $dy^\alpha \in \{x^i, \theta^\mu\}$.
\end{definition}