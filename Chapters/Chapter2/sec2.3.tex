\section{Super and Graded Vector Fields}
\label{sec:Sup_vector_fields}

In this section we will introduce the last concepts in order to understand the notion of BV-BFV theory.
We will briefly see how to define vector fields on supermanifold and graded manifolds.
We will then make some remarks on the concept of differential forms and finally introduce the \emph{cohomological vector field}.

\begin{definition}
    A \emph{supervector field} $X$ is a vector field on a supermanifold $\mathcal{M}$, which in local coordinates $(x^i, \theta^\mu)$ can be written as
    \begin{equation*}
        X = \sum_{i, \mu} X^i \partial_{x^i} + X^\mu \partial_{\theta^\mu} .
    \end{equation*}
\end{definition}

\begin{definition}
    A \emph{graded vector field} $X$ of degree $k$ on a graded manifold $\mathcal{M}$ is a graded linear map $X: C^\infty(\mathcal{M}) \rightarrow C^\infty(\mathcal{M})[k]$, which satisfies the Liebniz rule
    \begin{equation*}
        X(fg) = X(f)g + (-1)^{k|f|} f X(g)
    \end{equation*}
    for two functions $f,g \in C^\infty(\mathcal{M})$.
\end{definition}

We will use a $(+1)$-vector field $Q: C^\infty(\mathcal{M}) \rightarrow C^\infty(\mathcal{M})[1]$ as a differential on the manifold $\mathcal{M}$.
Furthermore, we will require the vector field $Q$ to be \emph{cohomological}, i.e. $Q^2 = 0$.
In particular, this will be used as our de Rham differential.

\begin{definition}
    A \emph{graded symplectic form} $\omega$ of degree $k$ on a graded manifold $\mathcal{M}$ is a closed, non-degenerate $2$-form
    \begin{equation*}
        \omega : T\mathcal{M} \rightarrow T^*[k]\mathcal{M}
    \end{equation*}
    whch can be written in local coordinates as
    \begin{equation*}
        \omega = \sum_{i,j} dy^i \omega_{ij} dy^j
    \end{equation*}
    with $dy^\alpha \in \{x^i, \theta^\mu\}$.
\end{definition}