\section{Symplectic Geometry}
\label{sec:symplectic_geom}

In this section we briefly introduce the concepts of symplectic forms on manifolds, Hamiltonian and symplectic vector fields.

Let $\omega$ be a $2$-form on a manifold $M$. Note that the map
\begin{equation*}
    \omega_q : T_q M \wedge T_q M \rightarrow \mathbb{R}
\end{equation*}
is skew-symmetric, meaning that
\begin{equation*}
    \omega(v, w) = -\omega(w, v), \quad \forall v, w \in T_q M.
\end{equation*}

If $\omega$ is skew-symmetric and non-degenerate, then is called a \emph{symplectic form}.

\begin{definition}
    A 2-form $\omega \in \Omega^2(M)$ is called \emph{symplectic} if $\omega$ is closed and $\omega_q$ is symplectic for all $q \in M$.
\end{definition}

It directly follows the definition of symplectic manifold.
\begin{definition}
\label{def:sympl_mnf}
    A pair $(M, \omega)$, where $M$ is a manifold and $\omega$ is a symplectic form, is called a \emph{symplectic manifold}.
\end{definition}

Recall now that the contraction with a vector field $X \in \mathfrak{X}(M)$ of a $2$-form is a map

\begin{equation*}
    \iota_X : \Omega^2(M) \rightarrow \Omega^1(M).
\end{equation*}

If we let $M$ be a symplectic manifold with a symplectic form $\omega$, then by non-degeneracy of $\omega$ it is possible to find a unique vector field $X_H$ such that its contraction with $\omega$ is equal to the de Rham differential of a particular smooth function $H \in C^\infty(M)$, called Hamiltonian function.

\begin{definition}
    Let $(M, \omega)$ be a symplectic manifold and $H:M\rightarrow \mathbb{R}$ a smooth function.
    A vector field $X_H \in \mathfrak{X}(M)$ satisfying
    \begin{equation*}
        \iota_{X_H}\omega = \dd H
    \end{equation*}
    is called \emph{Hamiltonian vector field} with \emph{Hamiltonian function} $H$.
\end{definition}

In particular, since the $1$-form $\iota_{X_H} \omega$ is the de Rham differential of a function $H$, this means that $\iota_{X_H} \omega$ is exact.
On the other hand, if $\iota_X \omega$ is closed, using Cartan's magic formula, we can compute the Lie derivative

\begin{equation*}
    L_X \omega = \iota_X \dd \omega + \dd \iota_X \omega = 0,
\end{equation*}
since $\omega$ is closed for being a symplectic form.

\begin{definition}
    A vector field $X$ on a symplectic manifold $(M, \omega)$ such that $L_X \omega = 0$ is called \emph{symplectic vector field}.
\end{definition}

Since, by definition, a symplectic manifold is endowed with a symplectic form $\omega$, it is then possible to define the Poisson bracket on the manifold.
\begin{definition}
    The \emph{Poisson bracket} of two functions $f, g \in C^\infty(M)$ on a symplectic manifold $(M, \omega)$ is defined by
    \begin{equation*}
        \{f, g\} \coloneqq \omega(X_f, X_g) .
    \end{equation*}
\end{definition}
In the definition above, the notation $X_f$ denotes the vector field associated with $f$ such that $\iota_{X_f} \omega = \dd f$ holds.
Thus, it is the Hamiltonian vector field associated to the Hamiltonian function $f$ (the same applies for the function $g$).